\newpage
\section{Preliminares}
Los fenómenos de transporte son todos los procesos irreversibles
de naturaleza estadística derivados de los movimientos aleatorios
continuo de las moléculas, principalmente observadas en los
fluidos. Este se basa en 2 conceptos primarios:

\begin{enumerate}
    \item \textit{Ecuaciones Constitutivas} que son una relación
    entre las variables termodinámicas o mecánicas de un
    sistema físico: presión, volumen, tensión, etc.
    \item \textit{Leyes de conservación} i.e. 
    que en la evolución temporal de un sistema aislado ciertas
    magnitudes permanecen constantes.
\end{enumerate}

Algunos ejemplos destacados de estos son \textit{ley de conducción
de calor de Fourier}que describe el flujo de calor o gradientes
de temperatura y las \textit{Ecuaciones de Navier-Stokes} que
describe el flujo de un fluido. Estas ecuaciones también
demuestran la conexión profunda entre los fenómenos de transporte
y la termodinámica, una conexión que explica por qué los fenómenos
de transporte son irreversibles.

Casi todos estos fenómenos físicos involucran sistemas que buscan
su estado de energía más bajo de acuerdo con el principio de
energía mínima, momento en el que ya no hay fuerzas impulsoras
en el sistema y el transporte cesa. 
la transferencia de calor es el intento del sistema por lograr el
equilibrio térmico con su entorno, al igual que el transporte de
masa y de momento mueve el sistema hacia el equilibrio químico y
mecánico.

Los ejemplos de procesos de transporte incluyen la conducción de
calor (transferencia de energía), el flujo de fluido (transferencia
de momento), la difusión molecular (transferencia de masa),
la radiación y la transferencia de carga eléctrica en los
semiconductores.

\subsection{Masa molar}
\textit{Un mol es la cantidad
de sustancia que contiene tantas entidades elementales (átomos, moléculas, iones, electrones, etc)
como átomos hay en 0.012 kg de carbono 12}. También necesitamos
el número de Avogadro $N_A$ que es el siguiente

\begin{equation*}
    N_A = 6.02214179\times 10^{23} \unit{moléculas/mol}
\end{equation*}

Con esto definimos la maso molar $M$ de un compuesto,
esta es la masa de un mol, está será la masa $m_m$ de Una sola molécula
multiplicada por su número másico

\begin{equation}
    M = N_A m_m
\end{equation}

Así la masa $m$ estará dada por $m = M n$, donde $n$
es el número de moles.