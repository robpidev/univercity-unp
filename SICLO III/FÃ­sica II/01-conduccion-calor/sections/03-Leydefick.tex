\subsection{Transferencia de materia}
Se entiende por transferencia de materia a la tendencia de los
componentes de una mezcla a desplazarse desde una región de concentración
elevada a otra baja concentración. (~\cite{rod:2017})

\subsection{Difusión}
La difusión está caracterizada por el flujo
de difusión $J$ de un componente, esto es, por la
cantidad de materia que pasa en la unidad
de tiempo a través de una superficie dada en la dirección
normal a la superficie.
La densidad de flujo de difusión $j$
se puede definir como la cantidad de sustancia que
pasa en la unidad de tiempo a través de la unidad de área $S$
de la superficie dada en dirección normal a esta superficie.
asi se tiene

\begin{equation}
    \label{eq:densidad:flujo}
    j = \frac{dJ}{dS}
\end{equation}

por lo que tenemos que
\begin{equation}
    J = \int_S jDs
\end{equation}

si $j$ es constante entonces $J$ será
\begin{equation}
    J = jS
\end{equation}

\subsection{Fundamentemos de la difusión Molecular}
\begin{enumerate}
    \item Difusión es el mecanismo por e cual se produce el movimiento,
    debido a un estimulo físico, de un componente a través de una 
    mezcla.

    \item La principal causa de la difusión es la existencia de un gradiente de 
    concentración del componente que difunde. El gradiente de concentración provoca
    el movimiento del componente de una dirección tal que tiene a igualar las concentraciones y reducir el gradiente.
\end{enumerate}

\subsection{Difusión Molecular}
\begin{enumerate}
    \item Se produce por el movimiento de las moléculas individuales, debido a su energía térmica.
    \item El número de colisiones entre partículas es mayor en la zona de alta concentración, por lo que se da un flujo hacia la de menor concentración.
\end{enumerate}

\subsection{sistema para el estudio de la difusión molecular}
En el sistema a considerar es la película gaseosa comprendida entre la superficie del líquido y la boca
del tubo. En película gaseosa, muy cerca a la superficie líquida, se puede tomar la concentración de la especie, 
A como la de equilibrio como el líquido, es decir, 
que es la relación entre la presión de vapor de A
a la temperatura del sistema y la presión total, suponiendo
que $A$ y $B$ forman una mezcla gaseosa ideal,
dentro dle recipiente el soluto $A$ difunde a través de $B$ estancado.

\subsection{Ley de Fick}
La densidad de flujo de difusión es un vector. Considerando que
el valor de una de sus componentes es positivo cuando ésta esté dirigida
hacia el sentido positivo del eje, y negativo en caso contrario.

Como la sustancia se traslada de los lugares de mayor concentración a los
de menor concentración, el signo de la componente del flujo en una dirección
será el contrario del que da la derivada de la concentración en esa
dirección $(\partial c/ \partial n)$. Si la concentración 
aumenta de izquierda a derecha, el flujo va hacia la izquierda y viceversa.
Además, si la concentración de la solución es uniforme $\partial c / \partial n = 0$,
no habrá flujo de difusión. Considerando todo esto, para
un sistema estacionario macroscópico de dos componentes, homogéneo en lo que
respecta la temperatura y presión, la densidad
de flujo de difusión de uno de los componentes, debido a difusión molecular, 
vide dada por la \textit{Ley de Fick}~\cite{agus:2011}

\begin{equation}
    j_i = -D \frac{\partial c_i}{\partial n}
\end{equation}

o utilizando la gradiente

\begin{equation}
    \vec \jmath = -D \vec \nabla c_i
\end{equation}


\begin{equation}
    J_A = D_{AB} \frac{-dC_A}{dz}
\end{equation}

donde $c_i$ es la concentración local de la sustancia, puede
medirse en masa por unidad de volumen, moles por unidad de volumen, etc.

\subsection{Difusión molecular en Estado estacionario}

De la ley de Fick se deduce la siguiente ecuación

\begin{equation}
    N_A = (N_A + N_B) \frac{C_A}{C_T} - D_{AB} \frac{dC_A}{dz}
\end{equation}

El primer sumando es lo que se mueve de $A$ debido al flujo global
del sistema.
El segundo sumando es la densidad de flujo que resulta
de la difusión.


\begin{itemize}
    \item $D_{AB}$: difusividad del compuesto $A$ en $B$
    \item $dC_A/dz$: Gradiente de concentración del compuesto $A$ en
    la dirección de $z$.
    \item $N_A$ es la densidad de flujo del compuesto $A$ con respecto a los ejes fijos.
    \item $N_B$: densidad de flujo dle compuesto $B$ con respecto a ejes fijos.
    \item $C_A$: Concentración molar del compuesto $A$
    \item $C_T$: Concentración molar total
\end{itemize}

\subsection{Difusividad}
\begin{itemize}
    \item Propiedad de transporte en función de la temperatura, presión
    y la naturaleza de los componentes
    \item Dimensione: $L^2/T$
    \item Se carece de datos de difusividad para 
    la mayor parte de las mezclas que tienen interés en ingeniería.
    Es preciso estimarlas a partir de correlaciones.
\end{itemize}