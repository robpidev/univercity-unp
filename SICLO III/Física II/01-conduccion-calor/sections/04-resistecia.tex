\subsection{Resistencia Térmica}
como se dedujo en~\ref{subsub:coduction} la Conducción
estaba dada por $R = L/k$, pero esto
se hizo para un elemento discreto (i.e.\ una sola capa
de material homogéneo y caras planoparalelas separas por
un espesor $L$).

Sin embargo esto no sirve si el elemento 
no es homogéneo, pero si su heterogeneidad se distribuye
uniformemente, en un laboratorio Se
puede obtener un coeficiente de conductividad
util $C$ como la media ponderada de los 
coeficientes de cada material, asi se puede
definir la Resistencia térmica $R$ como

\begin{equation}
    \label{eq:resistencia-het}
    R = \frac{1}{C}
\end{equation}

Si un elemento está formado por varias
capas con sus respectivas resistencia térmicas
$R_1 + R_2 + \cdots + R_n$,
entonces la resistencia total $R_T$ se puede
deducir que es

\begin{equation}
    \label{eq:resistencia-total}
    R_t = R_1 + R_2 + \cdots + R_n
\end{equation}

\subsubsection{Resistencia térmica superficial}
Al pasar el calor de un fluido a un elemento sólido (en general, del aire ambiente a un elemento constructivo) se produce una resistencia a este paso, que varía con la velocidad del fluido (velocidad del aire), rugosidad de la superficie, etc.
Y que se llama resistencia superficial. Tiene la misma ecuación dimensional que las resistencias de los elementos constructivos.

\subsubsection{Transmitancia térmica}
Es la cantidad de energía que atraviesa por 
unidad de tiempo, una superficie
de elemento conductivo de caras planas y paralelas
cuando entre dichas caras hay un gradiente térmico.
Es el inverso a la resistencia térmica

\begin{equation}
    \label{eq:transmitancia}
    U = \frac{1}{R}
\end{equation}

\subsubsection{Resistencia térmica total}
Cuando el elemento descrito está en una situación real, 
con aire ambiente en sus dos caras, se define la resistencia
térmica total $R_t$ es la suma de la resistencia
térmica del elemento conductivo más las
resistencias térmicas superficiales: es la inversa
de la transmitancia $U$, o 
coeficiente de transmisión de calor 
de un elemento $k$. De esto se deduce que

\begin{equation}
    R_t = 1/U = R_{s_e}+ R_{s_i} + R
\end{equation}

O si el elemento tiene varias capas, usando
la ecuación~\ref{eq:resistencia-total}, obtenemos

\begin{equation}
    \label{eq:resistencia:total}
    R_t = \frac{1}{U} = R_{s_e} + R_{s_i} + R_1 + R_2 + \cdots + R_n
\end{equation}

\subsection{Tiempo de igualación}
Considere un cuerpo de forma
arbitraria y masa $m$, volumen $V$, 
área superficial $A_s$, densidad $rho$ y calor
específico $C_p$, inicialmente a una temperatura $T_i$
. En el instante $t=0$, el cuerpo está colocado en un medio a La
temperatura $T_\infty$ y se lleva a efecto 
transferecia de calor entre ese cuerpo y su medio
ambiente, con un coeficiente de transferencia de calor
$h$. En beneficio de la discusión, se supondrá que $T_\infty > T_i$,
pero el análisis es igualmente válido para el caso opuesto.
Se supondrá que el análisis de sistemas concentrados es
aplicable, de modo que la temperatura permanece uniforme dentro del
cuerpo en todo momento y sólo cambia con el tiempo, $T = T(t)$.

Durante un intervalo diferencial de tiempo, $dt$, la temperatura
del cuerpo es elevada de una cantidad diferencial $DT$. Un
balance de energía del sólido para el intervalo de tiempo $dt$
se puede expresar como

\begin{equation}
    hA_s(T_\infty -T)dt = mC_p dT
\end{equation}

Como $m = \rho V$ y $dT - d(T - T_\infty)$,
puesto que $T_\infty$ es constante, remplazando
en la ecuación anterior

\begin{equation*}
    \frac{d(T-T_\infty)}{T - R_\infty}
        = -\frac{hA_s}{\rho V c_p} dt
\end{equation*}

Integrando desde $T = 0$, en el cual $T = T_i$, hasta
cualquier instante de $t$, en el cual $T = T(t)$, nos da

\begin{equation}
    \label{eq:enfrimaiento:tiempo}
    \ln\left(\frac{T(t) - T_\infty}{T_i - T_\infty}\right)
     = - \frac{hA_s}{\rho V c_p}t
\end{equation}

Si dejamos en función de $t$ tenemos
\begin{equation}
    \label{eq:enfriamiento:e}
    \ln\left(\frac{T(t) - T_\infty}{T_i - T_\infty}\right)
        = e^{-bt}
\end{equation}
donde
\begin{equation}
    b = \frac{hA_s}{\rho V c_p} (1/s)
\end{equation}

Esta ecuación nos permitirá calcular lo siguiente

\begin{enumerate}
    \item La ecuación~\ref{eq:enfriamiento:e} permite
    calcular la temperatura $T(t)$ de un cuerpo en el
    instante $t = 0$, de modo alternativo, el tiempo $t$
    requerido para alcanzar el específico $T(t)$.
    \item La temperatura de un cuerpo se aproxima a la del medio ambiente, $T_\infty$, en
    forma exponencial. Al principio, la temperatura del cuerpo cambia con rapidez pero, posteriormente, lo hace más bien con lentitud. Un valor grande de b indica que el cuerpo tenderá a alcanzar la temperatura del medio
    ambiente en un tiempo corto. Entre mayor sea el valor del exponente $b$,
    mayor será la velocidad de decaimiento de la temperatura. Note que $b$ es
    proporcional al área superficial, pero inversamente proporcional a la masa y al calor específico del cuerpo. Esto no es sorprendente, ya que tarda
    más tiempo en calentarse o enfriarse una masa grande, en especial cuando tiene un calor específico grande
\end{enumerate}

Podemos remplazar $T(t)$ en la ecuación~\ref{eq:enfriamiento-newton}, 
con lo que nos queda

\begin{equation}
    \label{eq:enf:t}
    q(t) = hA[T(t) - T_\infty]
\end{equation}

Por lo que la cantidad total de la transferencia
de calor entre el cuerpo y el medio circundante
durante el intervalo $t = 0$ hasta $t$ es simplemente
el cambio en el contenido de energía de ese cuerpo.
Así la ecuación será

$Q = mc_p[T(t) - T_i]$

La cantidad de transferencia de calor llega a su límite superior
cuando el cuerpo alcanza la temperatura $T_\infty$ del medio
circundante. Por lo tanto, la transferencia de calor máxima
entre el cuerpo y sus alrededores es:

$Q = mc_p(T_\infty - T_i)$
