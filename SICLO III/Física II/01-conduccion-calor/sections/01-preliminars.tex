\subsection{Cantidad de calor}
Si introducimos una cuchara fría en una taza
con agua caliente, la cuchara tenderá a  calentarse
y el agua tenderá a enfriarse, hasta encontrar
una temperatura de equilibrio, si generalizamos esto,
ya que pasa en la naturaleza constantemente, por lo que nace
la necesidad para poder estudiar este  fenómeno matemáticamente 
la necesidad de medir esto, la cual llamaremos
cantidad de calor $Q$ o simplemente calor. Para esto primero se define
la unidad de medida del calor la cual es la caloría.
una caloría se define como \textit{la cantidad de calor
que se necesita para elevar 1 g de agua de 14.5 a 15.5 °C}, 
como para subir la temperatura necesitamos energía en forma 
de calor, entonces la caloría también es calor, por lo que
de los datos experimentales se tiene

\begin{align*}
    1\unit{cal} = 4.186 \text{ J}\\
    1 \unit{kcal} = 4186 \text{ J}
\end{align*}

\subsection{Calor y Calor especifico}
La cantidad de calor $Q$ que se necesita para elevar una masa $m$
de una temperatura $T_0$ a $T_1$ depende directamente de la masa
pues de las observaciones experimentales lo an demostrado, 
así también como del tipo de material y la variación de la temperatura
por lo que se tiene que

\begin{equation}
    \label{eq:calor}
    Q = mc\Delta T, \text{ siendo } \Delta T = T_1 - T_0
\end{equation}

Para un cambio infinitesimal de temperatura $dT$ y calor
correspondiente $dT$ se obtiene la ecuación siguiente

\begin{equation}
    \Rightarrow dQ = mc\Delta T
\end{equation}

Así el $C$ llamado el calor especifico es
\begin{equation}
    c = \frac{1}{m}\frac{dQ}{dT}
\end{equation}

De esto se deduce que el calor especifico del
agua es
\begin{align*}
    c = 1 \unit{cal/g°C} \text{ o }    c = 4186 \unit{J/kg°C}
\end{align*}

\subsection{Capacidad calorífica molar}
Primero definamos un mol, \textit{Un mol es la cantidad
de sustancia que contiene tantas entidades elementales (átomos, moléculas, iones, electrones, etc)
como átomos hay en 0.012 kg de carbono 12}. También necesitamos
el número de Avogadro $N_A$ que es el siguiente

\begin{equation*}
    N_A = 6.02214179\times 10^{23} \unit{moléculas/mol}
\end{equation*}

Con esto definimos la maso molar $M$ de un compuesto,
esta es la masa de un mol, está será la masa $m_m$ de Una sola molécula
multiplicada por su número másico

\begin{equation}
    M = N_A m_m
\end{equation}

Así la masa $m$ estará dada por $m = M n$, donde $n$
es el número de moles, remplazando en la ecuación~\ref{eq:calor}
obtenemos

\begin{equation}
    Q = nMc\Delta T
\end{equation}

Al número $C = Mc$ se le llama la capacidad calorífica molar,
Asi el calor es

\begin{equation}
    \label{eq:calor-molar}
    Q = nC\Delta T
\end{equation}

Por lo que para una variación infinitesimal de 
temperatura $dT$ es

\begin{equation*}
    dQ = nCdT
\end{equation*}

Así la capacidad calorífica molar queda
\begin{equation}
    \label{eq:calor-especifico-molar}
    C = \frac{1}{n}\frac{dQ}{dT} = Mc
\end{equation}

\subsection{Mecanismos de transferencia de calor}
Hay diferentes formas como el calor se transfiere
entre dos puntos, ademas de que este depende del medio 
por el cual se transfiere, dependiendo 
del material, estos pueden
ser \textbf{aislantes:} \textit{Los que no transfieren
calor} \textbf{conductores: } \textit{Los que si transfieren
el calor}. Se a clasificado de la siguiente manera


\subsubsection{Conducción}\label{subsub:coduction}
Es la que se da por un medio a traves de los materiales conductores, de 
las observaciones experimentales se a obtenido lo 
siguiente, la transferencia de calor $dQ$ a traves de una
barilla conductora en un tiempo determinado $dt$ es directamente proporcional a la longitud $L$ y sección transversal
$A$ de la varilla, multiplicado por $\Delta T = T_B - T_A$, donde
$T_A$ y $T_B$ son las temperaturas de los extremos de la barilla,
ademas que esta depende de la conductividad térmica $k$ del 
material, i.e.

\begin{equation}
    \label{eq:conduccion}
    H = \frac{dQ}{dt} = kAL\frac{T_A - T_B}{L}
\end{equation}

si la temperatura varia
de manera no uniforme a traves 
de la varilla podemos
asignar la coordenada $x$ así introduciendo
la gradiente de $dT/dx$ temperatura se tiene:

\begin{equation}
    \label{eq:conduccion-x}
    H = \frac{dQ}{dt} = \frac{-kA}{L}\frac{dT}{dx}
\end{equation}

Ademas podemos asignar $R = L/(kA)$, siendo $R$ la resistividad
térmica, con lo que la ecuación quedará

\begin{equation}
    \label{eq:conduccion-r}
    H = \frac{dQ}{dt} = -\frac{1}{R}\frac{dT}{dx}
\end{equation}

\subsubsection{Convección}
La convección es transferencia de calor por movimiento de una masa de fluido de una
región del espacio a otra.

La transferencia de calor por convección es un proceso muy complejo, y no puede
describirse con una ecuación sencilla. Veamos algunos hechos experimentales:

\begin{enumerate}
    \item  La corriente de calor causada por convección es directamente proporcional al
    área superficial. Esto explica las áreas superficiales grandes de los radiadores y
    las aletas de enfriamiento.

    \item La viscosidad de los fluidos frena la convección natural cerca de una superficie
    estacionaria, formando una película superficial que, en una superficie vertical,
    suele tener el mismo valor aislante que tiene 1.3 cm de madera terciada (valor
    R = 0.7). La convección forzada reduce el espesor de esta película, aumentando
    la tasa de transferencia de calor. Esto explica el “factor de congelación”: nos
    enfriamos más rápidamente en un viento frío que en aire tranquilo a la misma
    temperatura.

    \item a corriente de calor causada por convección es aproximadamente proporcional 5/4 
    nal a la potencia de la diferencia de temperatura entre la superficie y el cuerpo
    principal del fluido.

    De estas observaciones se puede en un principio 
    deducir la ley de enfriamiento de Newton

    \begin{equation}
        \label{eq:enfriamiento-newton}
        q = hA(T_w - T_\infty)
    \end{equation}

    donde $q$ es la rapidez de transferencia del calor,
    que esta relacionado directamente proporcional a $h$ llamado
    el coeficiente de transferencia por convección y $A$ el área
    de la superficie de contacto entre el fluido y la superficiales,
    ademas $T_w$ es la temperatura de la superficie y $T_\infty$,
    la temperatura suficientemente lejos de la superficie.
\end{enumerate}

\subsubsection{Radiación}
a corriente de calor causada por convección es aproximadamente proporcional a la potencia de la diferencia de temperatura entre la superficie y el cuerpo
principal del fluido.