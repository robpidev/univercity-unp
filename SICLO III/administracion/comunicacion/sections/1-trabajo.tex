\newpage
\section{Trabajo en equipos.}
Para que las organizaciones logren funcionar en el contexto actual, 
deben lograr un verdadero cambio cultural. Para ello, es fundamental
conformar equipos de trabajo exitosos con profesionales que
cuenten con las habilidades necesarias para sacar adelante los
objetivos y metas de la organización.

Uno de los pilares más importantes de cualquier organización son sus
equipos de trabajo, y la calidad de la colaboración que exista entre
ellos será fundamental para enfrentar el cambio. Sea cual sea
nuestro rol en el equipo, tenemos que tener claro que es muy difícil
tener un buen nivel de colaboración y desempeño desde el principio y,
por lo tanto, el manejo de expectativas de todos los miembros es muy
relevante.

Aunque pueda parecer evidente, cuando trabajamos en un equipo,
muchas veces se nos olvida que además de realizar nuestras tareas
individuales, para alcanzar nuestros objetivos debemos trabajar en
conjunto con otras personas, lo que implica adaptarnos a todas las
virtudes y complejidades que cada uno de nosotros tiene.

No hay que confundir entre grupo de trabajo y equipo de trabajo
ya que existen ciertas diferencias entres las definiciones de
Grupo y Equipo. Un grupo es algo más general, y un equipo es solo
un tipo especial de grupo.

\subsection{Grupo de trabajo}
este está formado por dos o más individuos que
entran en contacto \textit{personal} y \textit{significativo} en
forma continua. Se puede clasificar los grupo como formales o
informales

\begin{enumerate}
    \item un \textbf{Grupo formal} es el responsable de hacer el
    trabajo básico de una organización. Por ejemplo: Una división,
    un departamento o una unidad de negocio.

    \item un \textbf{grupo informal} consiste en un número pequeño
    de individuos que con frecuencia participan juntos en
    actividades y comparten sentimientos con el propósito de
    satisfacer sus necesidades mutuas. Estas actividades pueden
    apoyar, oponerse o no tener relación con las metas,
    los objetivos organizativos, las reglas o la autoridad superior
    de la organización. Un ejemplo de este subgrupo puede ser un
    grupo informal de empleados que, dentro de la empresa (aunque
    fuera de su horario laboral) busca solucionar problemas
    medioambientales o colaborar con ONG's o, simplemente, buscan
    divertirse y estrechar lazos de amistad.
\end{enumerate}

Los grupos de trabajo se caracterizan por
\begin{enumerate}
    \item \textbf{Las interacciones: } cada miembro se relaciona con los
    demás componentes del grupo de forma directa, sin intermediarios.
    
    \item \textbf{El surgimiento de normas: } el grupo crea unas normas
    de conducta muchas veces implícitas que regulan los comportamientos
    de los grupos.

    \item \textbf{La existencia de objetivos colectivos comunes}
    cada grupo existe y se da a sí mismo una razón de ser, una meta,
    unos objetivos que justifiquen su existencia, y orienta su energía
    a alcanzarlos.

    \item \textbf{La existencia de emociones y de sentimientos
    colectivos}
    los miembros se sienten identificados con una especie de
    “espíritu o alma de grupo”.

    \item \textbf{La emergencia de una estructura informal}
    los miembros del grupo tienden a estar especializados en
    determinadas funciones grupales.
\end{enumerate}

\subsection{Equipo de trabajo}
este consiste en un número pequeño de empleados con habilidades
complementarias que colaboran en un proyecto, están comprometidos
con un propósito común y son responsables de manera conjunta de
realizar tareas que contribuyan al logro de las metas de una
organización. Pueden llamarse de múltiples maneras: equipos
autorizados, grupos de trabajo autónomo, cuadrilla, equipos
auto-gestionados, equipos inter-funcionales, círculos de calidad,
equipos de proyectos, fuerzas de tarea, equipos de alto rendimiento,
equipos de respuesta de emergencia, comités, consejos

\subsubsection{Beneficios}
Los equipos de trabajo dirigidos eficazmente, hacen posible que las
empresas logren importantes objetivos de negocio estratégicos, que
podrían conducir a obtener ventajas competitivas en el mercado.
Entre los beneficios del uso de equipos de trabajo se incluyen:
Bajos costes y alta productividad, mejoras en la calidad, rapidez e
innovación.

\subsubsection{Tipos de equipos de trabajo}
Las metas estratégicas que serán logradas por los equipos de
trabajo: innovación, velocidad, reducciones de costos y calidad,
pueden ser muy parecidas. Sin embargo, las metas específicas de 
los equipos de trabajo con frecuencia difieren ampliamente. Para
diferentes propósitos se utilizan diferentes tipos de equipos de 
trabajo. Los equipos de trabajo para la solución de problemas,
funcionales, multidisciplinarios, autogestionados y virtuales son
cinco tipos comunes de equipos de trabajo, y cada uno tiene un
propósito diferente.

\begin{enumerate}
    \item \textbf{Equipos de trabajo para la solución de
    problemas.} Consta de empleados de diferentes áreas
    cuya meta es considerar cómo algo puede hacerse mejor.
    \begin{enumerate}
        \item \textbf{Círculos de calidad.} Grupo de
        empleados que se reúnen de manera regular para
        identificar, analizar y proponer soluciones para
        diversos tipos de problemas de producción.
        \item \textbf{Fuerza de tarea.} Un equipo que se
        forma para lograr una meta específica muy
        importante para una organización. 
    \end{enumerate}
    
    \item \textbf{Equipos de trabajo funcionales.}
    Incluye miembros de un sólo departamento que tienen la
    meta común de considerar asuntos y resolver problemas
    dentro de su área de responsabilidad y experiencia.
    
    \item \textbf{Equipos de trabajo multidisciplinarios
    (multifunctional).} Empleados de diversas áreas
    funcionales en ocasiones varios niveles de la
    organización que trabajan en forma colectiva en
    tareas específicas.
    
    \item \textbf{Equipos de trabajo autogestionados.}
    Empleados que tienen plena responsabilidad y autoridad
    para trabajar juntos diariamente para hacer un producto
    o entregar un servicio. El área de decisiones de los
    equipos autogestionados es amplia. Pueden:

    \begin{itemize}
        \item Participar en la selección de nuevos miembros.
        \item Programar el trabajo y las vacaciones de los miembros.
        \item Rotar las tareas y asignaciones de trabajo
        entre miembros.
        \item Ordenar materiales
        \item Capacitar a los nuevos miembros
        \item Ordenar materiales.
        \item Capacitar a los nuevos miembros.
        \item Decidir sobre el liderazgo del equipo.
        \item Proporcionar retroalimentación a los miembros
        del equipo.
        \item Establecer metas de desempeño.
        \item Evaluar el rendimiento del equipo.
        \item Vigilar el progreso hacia las metas del equipo.
        \item Comunicarse directamente con los proveedores y
        clientes.
        \item Diseñar proceso de trabajo, etc.
    \end{itemize}
    
    \item \textbf{Equipos de trabajo virtuales} Cumple y hace sus
    tareas sin que nadie esté físicamente presente en el mismo lugar
    o incluso al mismo tiempo. Puede haber encuentros ocasionales
    cara a cara aunque lo normal es que los equipos de trabajo
    virtuales se comuniquen vía e-mail, groupware, correo de voz,
    videoconferencia, redes sociales y otras tecnologías de la
    información y la comunicación. Los grupos de trabajos virtuales
    pueden ser funcionales, para la solución de problemas,
    mutidisciplinares y autogestionados. 
\end{enumerate}

Los equipos de trabajo se caracterizan por lo siguiente
\begin{enumerate}
    \item \textbf{Compromiso} Cada uno se compromete a esforzarse en
    en realizar y terminar las tareas designadas.
    \item \textbf{Complementariedad} Cada uno domina un área y tiene
    unas habilidades específicas que se complementan con las de los
    demás.
    \item \textbf{Comunicación} Es vital la comunicación abierta
    entre todos (ideas, compartir, info, avances, incidencias, errores,
    etc.)
    \item  \textbf{Confianza} Cada un confía en que los demás harán
    sus tareas y que todos juntos lograrán los objetivos marcados.
    \item \textbf{Coordinación} Se planifica y organiza todo el
    trabajo pra lograr avanzar y terminar las tareas en plazo y forma.
\end{enumerate}

\subsection{Creación de un equipo de trabajo}
Crear un equipo requiere tiempo, pues sus miembros atraviesan diferentes
etapas mientras pasan de ser un montón de extraños a un grupo unido
con metas en común.
Para conformar un equipo de trabajo, el grupo inicial debe
evolucionar desde su constitución hasta llegar a las siguientes
características.

\begin{enumerate}
    \item Objetivos comunes y acordados.
    \item Tares definidas y negociadas.
    \item Procedimientos explícitos.
    \item Buenas relaciones interpersonales.
    \item Alto grado de interdependencia.
\end{enumerate}


\subsubsection{Faces para la creación de un equipo de trabajo}
El modelo de desarrollo de equipos de \textit{Bruce Tuckman} contempla 5
etapas en las que se muestra la secuencia de la conformación de equipo de
trabajo:

\begin{enumerate}
    \item \textbf{Formación (Forming)}
    Es la primera etapa en la formación de un equipo, en la que cada uno
    de los miembros se vuelve consciente de los otros pero la confianza
    aún no está desarrollada, por lo que es difícil abordar y tomar
    decisiones. El líder del proyecto es necesario para dirigir al
    equipo y dar a conocer los objetivos del proyecto, así como también
    para distribuir los roles de los integrantes. Una buena forma de
    iniciar la primera reunión de cualquier equipo de trabajo es
    saludando a las personas que conoces y presentándote a las que no.
    Es una dinámica que no tomará mucho tiempo, pero permitirá la
    creación de los primeros vínculos entre el equipo, logrando así
    bajar la tensión.

    \item \textbf{Enfrentamiento / Conflicto (Storming)}
    El equipo comienza a trabajar hacia una actualización de los objetivos
    del proyecto. Aquí es donde los conflictos empiezan debido a la
    divergencia de opiniones y a la competencia. El rol del líder del
    proyecto se vuelve importante en esta etapa para detectar los
    conflictos y actuar de mediador si fuese necesario.

    \item \textbf{Normalización (Norming)}
    La comunicación y confianza han aumentado entre los colaboradores,
    por lo que a partir de este punto y con mayor frecuencia son
    capaces de resolver sus conflictos, pues son más cooperadores para
    alcanzar los objetivos del proyecto. El desempeño del equipo mejora
    ahora que son capaces de reconocer quién es el más capaz para realizar
    determinadas tareas; y el líder del proyecto asume un rol de soporte,
    sin la necesidad de controlar todos los detalles del proyecto como en
    las etapas tempranas.

    \item \textbf{Desempeño (performing)}
    El desarrollo y la colaboración del equipo para alcanzar los
    objetivos del proyecto aumenta porque entiende su propósito y
    objetivos. Dada la autonomía que alcanza el equipo, el rol del líder
    se vuelve más de supervisión y soporte.

    \item \textbf{Finalización / Disolución (Adjournment)}
    Es la etapa menos explorada del modelo, pero es cuando el equipo
    termina su objetivo inicial y cierra sus tareas. Hay una sensación
    de felicidad por terminar, pero a la vez la pérdida de amistades
    puede generar algo de depresión. En esta etapa se vuelve a planificar
    y reflexionar sobre las labores cumplidas y las metas alcanzadas.
\end{enumerate}

\subsubsection{Sentimientos que a menudo influye 
en la efectividad y en la cohesión}
A lo largo de las etapas de desarrollo del equipo de trabajo, los
miembros del equipo experimentan una variedad de reacciones psicológicas,
emociones o sentimientos. Los sentimientos se refieren al clima emocional
de un grupo. Los cuatro sentimientos que es más probable que influyan son:

\begin{enumerate}
    \item \textbf{Confianza:} Los miembros tienen confianza entres sí.
    \item \textbf{Apertura:} los miembros están interesados en realidad en lo que otros tienen que decir.
    \item \textbf{Libertad:} los miembros hacen lo que hacen por un sentido de responsabilidad con el grupo, no debido a la presión de otros.
    \item \textbf{Interdependencia:} los miembros se coordinan y trabajan juntos para lograr metas comunes.
\end{enumerate}

Cuanto más presentes estén estos sentimientos, es más probable que el
equipo de trabajo sea efectivo y los miembros estén satisfechos.
Cuanto mayor sea el grado en que están presentes los cuatro sentimientos,
mayor será el nivel de cohesión del grupo.

La \textbf{cohesión} es la intensidad de los deseos de los miembros de permanecer
en el equipo y su compromiso con él. La cohesión no puede ser dictada por
gestores, líderes de equipo u otros; un equipo o grupo cohesivo puede
trabajar en forma efectiva a favor o en contra de las metas de la organización

\subsubsection{Normas de conducta}
son las reglas de comportamiento que son compartidas en forma amplia y aplicadas por los miembros del equipo de trabajo. 
Las normas pueden especificar:

\begin{itemize}
    \item Cuánto trabajo hacer
    \item Cómo deberían ser tratados los clientes
    \item La importancia de la alta calidad
    \item Cómo deberían vestir los miembros del equipo
    \item Qué clase de chistes son aceptables
    \item Cómo deberían sentirse los miembros acerca de la organización
    \item Cómo deberían abordar a sus gestores
\end{itemize}

Existen dos tipos de normas muy importantes:

\begin{enumerate}
    \item \textbf{Normas de desempeño}: Existe una norma de desempeño cuando se han cumplido
    tres criterios
    
    \begin{enumerate}
        \item Hay  un estándar de comportamiento apropiado para los miembros del equipo.
        \item Los miembros deben estar de acuerdo con el estándar.
        \item Los miembros deben percatarse de que el equipo apoya la norma particular a 
        Través del sistema de recompensas y castigos.
    \end{enumerate}

    Normas como éstas reducen las probabilidades de que un miembro del equipo sea
    un colado. Un colado es un miembro del equipo que no contribuye en forma plena
    al desempeño del equipo, pero aún así comparte las recompensas del equipo.
    
    \item \textbf{Normas para manejar los conflictos}
    Las normas concernientes a la forma de manejar los conflictos dentro del
    equipo son importantes para equipos que participan en gran cantidad de
    solución de problemas y toma de decisiones. Cuando las normas del equipo
    enfatizan en el conflicto, puede desarrollarse el pensamiento grupal.
    El pensamiento grupal es una mentalidad de acuerdo a cualquier costo,
    que produce toma de decisiones ineficaces del equipo de trabajo y puede
    conducir a malas soluciones. La probabilidad del pensamiento grupal
    aumenta cuando:

    \begin{itemize}
        \item La presión de los compañeros para someterse es grande.
        \item Un líder muy directivo presiona para una interpretación del problema y curso de acción particulares.
        \item Existe la necesidad de procesar un asunto complejo y poco estructura bajo condiciones de crisis.
        \item El grupo está aislado.
        \item Los miembros del equipo tienen antecedentes similares.
    \end{itemize}

    En lugar de conflicto agobiante, un mejor enfoque para manejar los
    desacuerdos es participar en una controversia productiva. La
    controversia productiva ocurre cuando los miembros del equipo
    valoran diferentes puntos de vista y buscan exteriorizarlos para
    facilitar la solución de problemas creativa. Para asegura la
    controversia constructiva, los miembros del equipo de trabajo
    deben establecer reglas básica para mantenerlos enfocados en
    los asuntos, en lugar de las personas y diferir las decisiones
    hasta que ya se hayan explorado varios asuntos e ideas.
\end{enumerate}

\subsection{Ventajas del trabajo en equipo.}
La actividad del grupo produce un resultado que excede a la contribución de cualquier miembro
tomado individualmente y a la suma de todos ellos. Este hecho se conoce como SINERGIA, un
término muy utilizado en medicina donde representa el efecto adicional que dos órganos
producen al trabajar asociados. Por tanto, la sinergia es la suma de energías individuales que
se multiplica progresivamente, reflejándose sobre la totalidad del grupo. Por eso se dice que en
un equipo de trabajo dos más dos no son cuatro, sino que puede ser cinco, nueve o quince.
\begin{equation*}
    2 + 2 = 5 \text{ o más}
\end{equation*}

Otras ventajas del trabajo en equipo son

\begin{enumerate}
    \item \textbf{Aumenta la motivación} de los participantes que tienen la oportunidad de aplicar sus
    conocimientos y competencias y ser reconocidos por ello, desarrollando un sentimiento de
    autoeficacia y pertenencia al grupo.

    \item \textbf{Mayor compromiso.} Participar en el análisis y toma de decisiones compromete con las
    metas del equipo y los objetivos organizacionales. Si se fomenta la participación en la toma
    de decisiones, los miembros se implican y aceptan en mayor grado las soluciones o
    decisiones adoptadas.

    \item \textbf{Mayor número de ideas.} Los equipos permiten manejar un mayor número de información,
    conocimientos y habilidades.

    \item \textbf{Más creatividad.} La creatividad se estimula con
    la combinación de los esfuerzos de los
    individuos, lo que ayuda a generar nuevos
    caminos para el pensamiento y la reflexión
    sobre los problemas, procesos y sistemas, y con
    la diversidad de puntos de vista, lo que posibilita
    una perspectiva más amplia.

    \item \textbf{Mejora la comunicación.} Compartir ideas y
    puntos de vista con otros en un entorno que
    estimula la comunicación abierta y positiva,
    contribuye a mejorar el funcionamiento de la
    organización.
    
    \item \textbf{Mejores resultados.} Cuando las personas
    trabajan en equipo se proporciona mayor seguridad y confianza en las decisiones tomadas
    frente al carácter autocrático y arbitrario que se percibe en las decisiones individuales.

    \item \textbf{Desarrollo de la identidad grupal.} El trabajo en equipo proporciona medios para
    desarrollar una “identidad grupal” que potencia el compromiso y la implicación de los
    miembros entre sí, en relación con la tarea y otros objetivos.
\end{enumerate}

\subsection{Causas del mal desempeño de un equipo de trabajo}
Cuando los equipos de trabajo no desempeñan su actividad tan bien como se
supone que deberían, puede deberse a muchas razones. En un principio, se
piensa en los procesos internos del equipo. Sin embargo, también puede
ser debido al sistema externo, que comprende condiciones e influencias
exteriores que existen antes y después de que se forme el equipo. El
sistema externo incluye
\subsubsection{Diseño del equipo}

\begin{itemize}
    \item \textbf{Tamaño del equipo}  Los equipos con más de una docena
    de integrantes tienen dificultades para comunicarse entre sí.
    Aumentar el tamaño el equipo causa los siguiente aspectos:

    \begin{enumerate}
        \item La demanda sobre el tiempo y la tensión del líder son mayores. El líder se vuelve más distante desde el punto de vista psicológico de los miembros del equipo.
        \item La tolerancia del equipo a la dirección del líder es mayor, y la toma de decisiones del equipo se vuelve más centralizada.
        \item La atmósfera del equipo es menos amigable, las comunicaciones son menos personales, se forman más camarillas dentro del equipo.
        \item Las reglas y procedimientos del equipo se vuelven más formalizadas.
        \item Aumenta la probabilidad de qu algunos miembros sean colados.
    \end{enumerate}
    
    Para la toma de decisiones innovadoras es probable que el tamaño real
    del equipo de trabajo sea entre 5 y 9 miembros. Si se requieren
    equipos más grandes, el uso de sub-equipos puede ser una buena
    alternativa. El líder de un equipo de trabajo grande necesita ser
    consciente de la posibilidad de que puedan formarse sub-equipos o
    camarillas.

    \item \textbf{Ubicación del equipo.} La ubicación del equipo se
    refiere a la proximidad de los miembros del equipo. Dos aspectos de
    la ubicación del equipo son importantes:
    \begin{enumerate}
        \item La proximidad con otros equpipos de trabajo y miembros
        de la organización y
        \item la proximidad de los miembros del equipo entre Sí.
    \end{enumerate}
    La proximidad ideal entre equipos depende del trabajo que se realiza.
    Cabe destacar que en los equipos virtuales, no existe una proximidad
    como tal.

    \item \textbf{Cultura}
    Las culturas sociales y de las organizaciones dentro de las cuales
    operan los equipos de trabajo son aspectos importantes del contexto externo. Las diferencias entre las culturas sociales, las diferencias en el lenguaje y una débil, pobre o inexistente cultura organizacional, pueden llevar al fracaso del equipo de trabajo.
    
    \item \textbf{Selección de los miembros.}
    Las características necesarias en un empleado que trabaja en
    aislamiento relativo son diferentes de las que se necesitan en un
    empleado que debe trabajar en un ambiente de equipo. En los equipos
    de trabajo, los rasgos de personalidad, de carisma y de
    escrupulosidad parecen ser especialmente importantes.
    Además, de las seis competencias de gestión, la comunicación y el
    trabajo en equipo son esenciales para trabajar en todo tipo de
    equipos de trabajo.

    \item \textbf{Capacitación del equipo}
    Para conseguir un correcto rendimiento del equipo de trabajo,
    es también muy importante na correcta y continua capacitación
    del equipo. Esta capacitación debe darse en las siguientes áreas:

    \begin{itemize}
        \item Capacitación en gestión y liderazgo
        \item Capacitación para gestionar las reuniones
        \item Capacitación para apoyar el desacuerdo
        \item Capacitación para comprometerse con una decisión de equipo
        \item Capacitación para utilizar tecnologías basadas en grupos. 
    \end{itemize}
\end{itemize}